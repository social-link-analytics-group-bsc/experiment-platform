\documentclass{article}

\usepackage{arxiv}

\usepackage[utf8]{inputenc} % allow utf-8 input
\usepackage[T1]{fontenc}    % use 8-bit T1 fonts
\usepackage{hyperref}       % hyperlinks
\usepackage{url}            % simple URL typesetting
\usepackage{booktabs}       % professional-quality tables
\usepackage{amsfonts}       % blackboard math symbols
\usepackage{nicefrac}       % compact symbols for 1/2, etc.
\usepackage{microtype}      % microtypography
\usepackage{lipsum}		% Can be removed after putting your text content
\usepackage{graphicx}
\usepackage{natbib}
\usepackage{doi}
\usepackage{csvsimple}
\usepackage{lscape}
\usepackage{mathtools}
\usepackage{csvsimple}
\usepackage{tabularx}

\usepackage{color, soul}
\newcommand{\TODO}{\hl}

\title{Results from experiments}

%\date{September 9, 1985}	% Here you can change the date presented in the paper title
%\date{} 					% Or removing it

\author{\hspace{1mm}
	\href{https://orcid.org/0000-0000-0000-0000}{\hspace{1mm}Olivier R.~Philippe} \\
	Department of Life Sciences\\
	Barcelona Supercomputing Center\\
	Barcelona, ES 08034 \\
	\texttt{olivier.philippe@bsc.es} \\
	%% examples of more authors
	\And
	\href{https://orcid.org/0000-0000-0000-0000}{\hspace{1mm}Nataly ~Buslón} \\
	Department of Life Sciences\\
	Barcelona Supercomputing Center\\
	Barcelona, ES 08034 \\
	\texttt{nataly.buslon@bsc.es} \\
	\And
	Diego ~Saby \\
    Department of Life Sciences\\
	Barcelona Supercomputing Center\\
	Barcelona, ES 08034 \\
	\texttt{diego.saby@bsc.es} \\
	\And
	\hspace{1mm}Javier ~del Valle \\
	Department of Life Sciences\\
	Barcelona Supercomputing Center\\
	Barcelona, ES 08034 \\
	\texttt{javier.delvalle@bsc.es} \\
	\And
	\hspace{1mm}Oriol ~Puig \\
	Department of Life Sciences\\
	Barcelona Supercomputing Center\\
	Barcelona, ES 08034 \\
	\texttt{oriol.puigrodriguez@bsc.es} \\
	\And
	\hspace{1mm}Ram\'on Salaverr\'ia  \\
	Departamento de Proyectos Period\'isticos\\
	Universidad de Navarra\\
	Pamplona, ES 31009 \\
	\texttt{rsalaver@unav.es} \\
	\And
	\hspace{1mm}Mar\'ia Jos\'e ~Rementeria \\
	Department of Life Sciences\\
	Barcelona Supercomputing Center\\
	Barcelona, ES 08034 \\
	\texttt{maria.rementeria@bsc.es} \\
}
% Uncomment to remove the date
%\date{}

% Uncomment to override  the `A preprint' in the header
%\renewcommand{\headeright}{Technical Report}
%\renewcommand{\undertitle}{Technical Report}
\renewcommand{\shorttitle}{\textit{arXiv} Template}

%%% Add PDF metadata to help others organize their library
%%% Once the PDF is generated, you can check the metadata with
%%% $ pdfinfo template.pdf
\hypersetup{
pdftitle={Results from experiment},
pdfsubject={q-bio.NC, q-bio.QM},
%pdfauthor={Olivier R.~Philippe, Elias D.~Striatum},
pdfkeywords={First keyword, Second keyword, More},
}

\begin{document}
\maketitle

\begin{abstract}
	\lipsum[1]
\end{abstract}


% keywords can be removed
\keywords{\TODO{First} keyword \and \TODO{Second keyword} \and \TODO{More}}


\section{Introduction}


\section{Methods}
\label{sec:methods}

\subsection{Design}
20 News talking about the COVID were selected and separated in 10 False News and 10 True News. To judge which news was considered as disinformation \TODO{Add the proper methodology here}
The list of the news can be found in \ref{tab:list_news}


\subsection{Procedure}
\label{ssec:procedure}

During an online experiment, hosted on Google form, participants were asked to judge if a news was legitmate or a misinformation. 

After they answered, a list of multiple choice questions asked them what were the reasons that led to the decision. The questions were different in case of the subject answered True \ref{table:q_after_true} or False \ref{table:q_after_false}.
In the case they considered the news as True, another series of multiple choices questions asked them about their intentions of actions \ref{table:q_actions}. 


% \begin{table}[htb!]
%     \centering
%     \csvautobooktabular{./tables/justification_fake_right.csv}
%     \caption{Justifications for recognising news as False}
%     \label{tab:q_after_false}
% \end{table}

% \begin{table}[htb!]
%     \centering
%     \csvautobooktabular{./tables/q_true.csv}
%     \caption{Justifications for recognising news as True}
%     \label{tab:q_after_true}
% \end{table}

% \begin{table}[htb!]
%     \centering
%     \csvautobooktabular{./tables/q_actions.csv}
%     \caption{Intentions of action after considering a news True}
%     \label{tab:q_actions}

% \end{table}

Each participants read a legitimate news and a misinformation news presented in random order. After that questions were asked about their age, gender, political affiliation, education, religion and technological level.

\subsection{Subjects}
\label{ssec:subjects}

From the total of subjects (N=1063), subjects that completed the survey under 2 minutes (N=51) and the ones that required more than 25 minutes (N=9) were considered as outliers and removed, leaving the final dataset of 840 subjects. See figure \ref{fig:participants_time_completion}.

\begin{figure}[hbt]
    \centering
    \includegraphics[width=8cm]{plots/time_completion.png}
    \caption{Descriptive statistics on time completion from participants}
    \label{fig:participants_time_completion}
\end{figure}


\subsection{Recoding the answers}
Some answers were recoded to facilitate further analysis.

\subsection{Results}
\label{ssec:results}

The design of the study implies a pseudo replication, all subject have been asked to answer to two news and only 20 news have been selected. To control for this effect, a General Linear Mixed Model has been fitted to the data using R and the libray glmr from the R packages lme4 \cite{bates_fitting_2015}

Overall, neither the condition of Misinformation or Legitimate news seems to have significant impact on the subject capacity to recognise the type of news ({$\chi^2= 2.54,  \textit{df}=1, N=1680, \rho=.11 $}; see Fig. \ref{fig:prop_right_wrong_type} and Table \ref{tab:prop_right_wrong_type}).

Random effect


\begin{figure}[hbt]
    \centering
    \includegraphics[width=8cm]{plots/right_wrong_per_type.png
    }
    \caption{Proportion of right and wrong answers for each type of news.}
    \label{fig:prop_right_wrong_type}
\end{figure}


\begin{table}[htb!]
    \centering
    \csvautobooktabular{./tables/type_news_answer_count.csv}
    \caption{Proportion of right and wrong answers for each type of news.}
    \label{tab:prop_right_wrong_type}
\end{table}

\appendix

\section{List of News}
\label{sec:list_news}


\csvreader[
  tabular=| p{10cm} | p{2.5cm} |p{4cm}|,
  table head= \hline News Title & Type of news & Medio digital de publicación\\ \hline,
  late after last line=\\\hline,
]{./tables/news_titles.csv}{}%
{\csvcoli & \csvcolii & \csvcoliii}

\bigskip


% \begin{aligned}
%     \begin{split}
%   \operatorname{answer}_{i}  &\sim \operatorname{Binomial}(n = 1, \operatorname{prob}_{\operatorname{answer} = \operatorname{Right}} = \widehat{P}) \\
%     \log\left[\frac{\hat{P}}{1 - \hat{P}} \right] &=\alpha_{j[i],k[i]} \\
%     \alpha_{j}  &\sim N \left(\gamma_{0}^{\alpha} + \gamma_{1}^{\alpha}(\operatorname{recode\_politics}_{\operatorname{Centro}}) + \\ \gamma_{2}^{\alpha}(\operatorname{recode\_politics}_{\operatorname{Derecha}})\ + \\ \gamma_{3}^{\alpha}(\operatorname{recode\_education}_{\operatorname{No\_university\_studies}})\ + \\ \gamma_{4}^{\alpha}(\operatorname{dm\_genero}_{\operatorname{Masculino}})\ + \\ \gamma_{5}^{\alpha}(\operatorname{recode\_age}_{\operatorname{35-54}}) + \gamma_{6}^{\alpha}(\operatorname{recode\_age}_{\operatorname{>55}}) + \gamma_{7}^{\alpha}(\operatorname{recode\_religion}_{\operatorname{Religious}})\ + \\ \gamma_{8}^{\alpha}(\operatorname{dm\_tecnologia}_{\operatorname{Media}})\ + \\ \gamma_{9}^{\alpha}(\operatorname{dm\_tecnologia}_{\operatorname{Avanzada}}) + \\ \gamma_{100}^{\alpha}(\operatorname{dm\_genero}_{\operatorname{Masculino}} \times \operatorname{dm\_tecnologia}_{\operatorname{Media}}) + \gamma_{111}^{\alpha}(\operatorname{dm\_genero}_{\operatorname{Masculino}} \times \operatorname{dm\_tecnologia}_{\operatorname{Avanzada}}), \sigma^2_{\alpha_{j}} \right)
%     \text{, for id_sondea j = 1,} \dots \text{,J} \\
%     \alpha_{k}  &\sim N \left(\gamma_{0}^{\alpha} + \gamma_{1}^{\alpha}(\operatorname{type\_news}_{\operatorname{fake\_news}}) + \gamma_{2}^{\alpha}(\operatorname{topic}_{\operatorname{Science}}) + \gamma_{3}^{\alpha}(\operatorname{topic}_{\operatorname{Technology}}) + \gamma_{4}^{\alpha}(\operatorname{recode\_religion}_{\operatorname{Religious}} \times \operatorname{topic}_{\operatorname{Science}}) + \gamma_{5}^{\alpha}(\operatorname{recode\_religion}_{\operatorname{Religious}} \times \operatorname{topic}_{\operatorname{Technology}}) + \gamma_{6}^{\alpha}(\operatorname{recode\_education}_{\operatorname{No\_university\_studies}} \times \operatorname{topic}_{\operatorname{Science}}) + \gamma_{7}^{\alpha}(\operatorname{recode\_education}_{\operatorname{No\_university\_studies}} \times \operatorname{topic}_{\operatorname{Technology}}) + \gamma_{8}^{\alpha}(\operatorname{recode\_education}_{\operatorname{No\_university\_studies}} \times \operatorname{type\_news}_{\operatorname{fake\_news}}) + \gamma_{9}^{\alpha}(\operatorname{recode\_politics}_{\operatorname{Centro}} \times \operatorname{type\_news}_{\operatorname{fake\_news}}) + \gamma_{10}^{\alpha}(\operatorname{recode\_politics}_{\operatorname{Derecha}} \times \operatorname{type\_news}_{\operatorname{fake\_news}}) + \gamma_{11}^{\alpha}(\operatorname{dm\_genero}_{\operatorname{Masculino}} \times \operatorname{type\_news}_{\operatorname{fake\_news}}), \sigma^2_{\alpha_{k}} \right)
%     \text{, for news_title k = 1,} \dots \text{,K}
%     \end{split}
% \end{aligned}

\begin{equation}
  \operatorname{answer}_{i}  \sim \operatorname{Binomial}(n = 1, \operatorname{prob}_{\operatorname{answer} = \operatorname{Right}} = \widehat{P})
\end{equation}

\begin{equation}
    \log\left[\frac{\hat{P}}{1 - \hat{P}} \right] =\alpha_{j[i],k[i]}
\end{equation}

\begin{equation}
    \begin{split}
    \MoveEqLeft
    \alpha_{j}  \sim N (\beta_{0}^{\alpha}\ + \\
    & \beta_{1}^{\alpha}(\operatorname{politics}\ + \\
    & \beta_{3}^{\alpha}(\operatorname{education}\ + \\
    & \beta{4}^{\alpha}(\operatorname{gender}\ + \\
    & \gamma_{5}^{\alpha}(\operatorname{recode\_age}_{\operatorname{35-54}}) + \gamma_{6}^{\alpha}(\operatorname{recode\_age}_{\operatorname{>55}}) + \gamma_{7}^{\alpha}(\operatorname{recode\_religion}_{\operatorname{Religious}})\ + \\
    & \gamma_{8}^{\alpha}(\operatorname{dm\_tecnologia}_{\operatorname{Media}})\ + \\
    & \gamma_{9}^{\alpha}(\operatorname{dm\_tecnologia}_{\operatorname{Avanzada}}) + \\
    & \gamma_{100}^{\alpha}(\operatorname{dm\_genero}_{\operatorname{Masculino}} \times \operatorname{dm\_tecnologia}_{\operatorname{Media}}) + \gamma_{111}^{\alpha}(\operatorname{dm\_genero}_{\operatorname{Masculino}} \times \operatorname{dm\_tecnologia}_{\operatorname{Avanzada}}), \sigma^2_{\alpha_{j}})
    % \text{, for id_sondea j = 1,} \dots \text{,J} \\
    \end{split}
\end{equation}

\begin{equation}
    \begin{split}
    \MoveEqLeft
    \alpha_{k}  \sim N (\gamma_{0}^{\alpha} + \gamma_{1}^{\alpha}(\operatorname{type\_news}_{\operatorname{fake\_news}}) + \gamma_{2}^{\alpha}(\operatorname{topic}_{\operatorname{Science}}) + \gamma_{3}^{\alpha}(\operatorname{topic}_{\operatorname{Technology}}) + \\
    & \gamma_{4}^{\alpha}(\operatorname{recode\_religion}_{\operatorname{Religious}} \times \operatorname{topic}_{\operatorname{Science}}) + \gamma_{5}^{\alpha}(\operatorname{recode\_religion}_{\operatorname{Religious}} \times \operatorname{topic}_{\operatorname{Technology}}) +\\
    & \gamma_{6}^{\alpha}(\operatorname{recode\_education}_{\operatorname{No\_university\_studies}} \times \operatorname{topic}_{\operatorname{Science}}) + \gamma_{7}^{\alpha}(\operatorname{recode\_education}_{\operatorname{No\_university\_studies}} \times \operatorname{topic}_{\operatorname{Technology}}) +\\
    & \gamma_{8}^{\alpha}(\operatorname{recode\_education}_{\operatorname{No\_university\_studies}} \times \operatorname{type\_news}_{\operatorname{fake\_news}}) + \gamma_{9}^{\alpha}(\operatorname{recode\_politics}_{\operatorname{Centro}} \times \operatorname{type\_news}_{\operatorname{fake\_news}}) + \\
    & \gamma_{10}^{\alpha}(\operatorname{recode\_politics}_{\operatorname{Derecha}} \times \operatorname{type\_news}_{\operatorname{fake\_news}}) + \gamma_{11}^{\alpha}(\operatorname{dm\_genero}_{\operatorname{Masculino}} \times \operatorname{type\_news}_{\operatorname{fake\_news}}), \sigma^2_{\alpha_{k}} )
    % \text{, for news_title k = 1,} \dots \text{,K}
\end{split}
\end{equation}

\bibliographystyle{unsrtnat}
\bibliography{./rrssalut_experiment}  %%% Uncomment this line and comment out the ``thebibliography'' section below to use the external .bib file (using bibtex) .


%%% Uncomment this section and comment out the \bibliography{references} line above to use inline references.
% \begin{thebibliography}{1}

% 	\bibitem{kour2014real}
% 	George Kour and Raid Saabne.
% 	\newblock Real-time segmentation of on-line handwritten arabic script.
% 	\newblock In {\em Frontiers in Handwriting Recognition (ICFHR), 2014 14th
% 			International Conference on}, pages 417--422. IEEE, 2014.

% 	\bibitem{kour2014fast}
% 	George Kour and Raid Saabne.
% 	\newblock Fast classification of handwritten on-line arabic characters.
% 	\newblock In {\em Soft Computing and Pattern Recognition (SoCPaR), 2014 6th
% 			International Conference of}, pages 312--318. IEEE, 2014.

% 	\bibitem{hadash2018estimate}
% 	Guy Hadash, Einat Kermany, Boaz Carmeli, Ofer Lavi, George Kour, and Alon
% 	Jacovi.
% 	\newblock Estimate and replace: A novel approach to integrating deep neural
% 	networks with existing applications.
% 	\newblock {\em arXiv preprint arXiv:1804.09028}, 2018.

% \end{thebibliography}


\end{document}
